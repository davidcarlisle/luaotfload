\luatexsuppresslongerror1%% sigh ...
%% Copyright (C) 2009-2014
%%
%%      by  Elie Roux      <elie.roux@telecom-bretagne.eu>
%%      and Khaled Hosny   <khaledhosny@eglug.org>
%%      and Philipp Gesang <philipp.gesang@alumni.uni-heidelberg.de>
%%
%% This file is part of Luaotfload.
%%
%%      Home:      https://github.com/lualatex/luaotfload
%%      Support:   <lualatex-dev@tug.org>.
%%
%% Luaotfload is under the GPL v2.0 (exactly) license.
%%
%% ----------------------------------------------------------------------------
%%
%% Luaotfload is free software; you can redistribute it and/or
%% modify it under the terms of the GNU General Public License
%% as published by the Free Software Foundation; version 2
%% of the License.
%%
%% Luaotfload is distributed in the hope that it will be useful,
%% but WITHOUT ANY WARRANTY; without even the implied warranty of
%% MERCHANTABILITY or FITNESS FOR A PARTICULAR PURPOSE. See the
%% GNU General Public License for more details.
%%
%% You should have received a copy of the GNU General Public License
%% along with Luaotfload; if not, see <http://www.gnu.org/licenses/>.
%%
%% ----------------------------------------------------------------------------
%%

\documentclass{ltxdoc}

\makeatletter

\usepackage {metalogo,multicol,mdwlist,fancyvrb,xspace}
\usepackage [x11names] {xcolor}

\def \primarycolor   {DodgerBlue4} %%-> rgb  16  78 139 | #104e8b
\def \secondarycolor {Goldenrod4}  %%-> rgb 139 105 200 | #8b6914

\usepackage[
    bookmarks=true,
   colorlinks=true,
    linkcolor=\primarycolor,
     urlcolor=\secondarycolor,
    citecolor=\primarycolor,
     pdftitle={The Luaotfload package},
   pdfsubject={OpenType layout system for Plain TeX and LaTeX},
    pdfauthor={Elie Roux & Khaled Hosny & Philipp Gesang},
  pdfkeywords={luatex, lualatex, unicode, opentype}
]{hyperref}

\usepackage {fontspec}
\usepackage {unicode-math}

\setmainfont[
% Numbers     = OldStyle, %% buggy with font cache
  Ligatures   = TeX,
  BoldFont    = {Linux Libertine O Bold},
  ItalicFont  = {Linux Libertine O Italic},
  SlantedFont = {Linux Libertine O Italic},
]{Linux Libertine O}
\setmonofont[Ligatures=TeX,Scale=MatchLowercase]{Liberation Mono}
%setsansfont[Ligatures=TeX]{Linux Biolinum O}
\setsansfont[Ligatures=TeX,Scale=MatchLowercase]{Iwona Medium}
%setmathfont{XITS Math}

\usepackage{hologo}

\newcommand\TEX      {\TeX\xspace}
\newcommand\LUA      {Lua\xspace}
\newcommand\PDFTEX   {pdf\TeX\xspace}
\newcommand\LUATEX   {Lua\TeX\xspace}
\newcommand\XETEX    {\XeTeX\xspace}
\newcommand\LATEX    {\LaTeX\xspace}
\newcommand\LUALATEX {Lua\LaTeX\xspace}
\newcommand\CONTEXT  {Con\TeX t\xspace}
\newcommand\OpenType {\identifier{Open\kern-.25ex Type}\xspace}

%% \groupedcommand, with some omissions taken from syst-aux.mkiv
\let \handlegroupnormalbefore \relax
\let \handlegroupnormalafter  \relax

\protected \def \handlegroupnormal #1#2{%
  \bgroup % 1
  \def \handlegroupbefore {#1}%
  \def \handlegroupafter  {#2}%
  \afterassignment \handlegroupnormalbefore
  \let \next =
}

\def \handlegroupnormalbefore {%
  \bgroup % 2
  \handlegroupbefore
  \bgroup % 3
  \aftergroup \handlegroupnormalafter%
}

\def \handlegroupnormalafter {%
  \handlegroupafter
  \egroup % 3
  \egroup % 2
}

\let \groupedcommand \handlegroupnormal %% only the two arg version

\def \definehighlight [#1][#2]{%
  \ifcsname #1\endcsname\else
    \expandafter\def\csname #1\endcsname{%
      \leavevmode
      \groupedcommand {#2}\empty%
    }
  \fi%
}

%% old, simplistic definition: obsolete now that we have
%% \groupedcommand
%\def\definehighlight[#1][#2]%
 %{\ifcsname #1\endcsname\else
    %\expandafter\def\csname #1\endcsname%
      %{\bgroup#2\csname #1_indeed\endcsname}
    %\expandafter\def\csname #1_indeed\endcsname##1%
      %{##1\egroup}%
  %\fi}

\def\restoreunderscore{\catcode`\_=12\relax}

\definehighlight         [fileent][\ttfamily\restoreunderscore]         %% files, dirs
\definehighlight        [texmacro][\sffamily\itshape\textbackslash]     %% cs
\definehighlight     [luafunction][\sffamily\itshape\restoreunderscore] %% lua identifiers
\definehighlight      [identifier][\sffamily]                           %% names
\definehighlight          [abbrev][\rmfamily\scshape]                   %% acronyms
\definehighlight        [emphasis][\rmfamily\slshape]                   %% level 1 emph

\definehighlight       [Largefont][\Large]                              %% font size
\definehighlight       [smallcaps][\sc]                                 %% font feature
\definehighlight [nonproportional][\tt]                                 %% font switch

\newcommand*\email[1]{\href{mailto:#1}{#1}}

\renewcommand\partname{Part}%% gets rid of the stupid “file” heading

\usepackage{syntax}%% bnf for font request syntax

\usepackage{titlesec}

\def\movecountertomargin#1{\llap{\rmfamily\upshape#1\hskip2em}}
\def\zeropoint{0pt}
\titleformat \part
             {\normalsize\rmfamily\bfseries}
             {\movecountertomargin\thepart} \zeropoint {}
\titleformat \section
             {\normalsize\rmfamily\scshape}
             {\movecountertomargin\thesection} \zeropoint {}
\titleformat \subsection
             {\small\rmfamily\itshape}
             {\movecountertomargin\thesubsection} \zeropoint {}
\titleformat \subsubsection
             {\normalsize\rmfamily\upshape}
             {\movecountertomargin\thesubsubsection} \zeropoint {}

\usepackage{tocloft}
\renewcommand \cftpartfont   {\rmfamily\upshape}
\renewcommand \cftsecfont    {\rmfamily\upshape}
\renewcommand \cftsubsecfont {\rmfamily\upshape}
\setlength \cftbeforepartskip {1ex}
\setlength \cftbeforesecskip  {1ex}

\VerbatimFootnotes

%%%%%%%%%%%%%%%%%%%%%%%%%%%%%%%%%%%%%%%%%%%%%%%%%%%%%%%%%%%%%%%%%%%%%%%%%%%%%%%
%% structurals
%%%%%%%%%%%%%%%%%%%%%%%%%%%%%%%%%%%%%%%%%%%%%%%%%%%%%%%%%%%%%%%%%%%%%%%%%%%%%%%

\def \definestructural #1{%
  \expandafter \let \csname end#1\endcsname \relax

  \expandafter \def \csname begin#1\endcsname {%
    \@ifnextchar[{\csname begin#1indeed\endcsname}
                 {\csname begin#1indeed\endcsname[]}%
  }

  \expandafter \def \csname begin#1indeed\endcsname [##1]##2{%
    \edef \first {##1}%
    \ifx \first \empty
      \csname #1\endcsname [##2]{##2}%
    \else
      \csname #1\endcsname [\first]{##2}%
    \fi
  }
}

\definestructural {section}
\definestructural {subsection}
\definestructural {subsubsection}

\def \fakesection #1{\section*{#1}}

%%%%%%%%%%%%%%%%%%%%%%%%%%%%%%%%%%%%%%%%%%%%%%%%%%%%%%%%%%%%%%%%%%%%%%%%%%%%%%%
%% inline verbatim
%%%%%%%%%%%%%%%%%%%%%%%%%%%%%%%%%%%%%%%%%%%%%%%%%%%%%%%%%%%%%%%%%%%%%%%%%%%%%%%

%% Context offers both \type{…} and \type<<…>>, but not an unbalanced
%% one that we could map directly onto Latex’s \verb|…|.

\usepackage {listings}
\lstset {
  basicstyle=\ttfamily,
}

%\let \inlinecode \lstinline
\protected \def \inlinecode {\lstinline}

%%%%%%%%%%%%%%%%%%%%%%%%%%%%%%%%%%%%%%%%%%%%%%%%%%%%%%%%%%%%%%%%%%%%%%%%%%%%%%%
%% codelistings; this sucks hard since we lack access to buffers
%%%%%%%%%%%%%%%%%%%%%%%%%%%%%%%%%%%%%%%%%%%%%%%%%%%%%%%%%%%%%%%%%%%%%%%%%%%%%%%

\newcount \othercatcode  \othercatcode 12
\newcount \activecatcode \othercatcode 13

\newluatexcatcodetable \vrbcatcodes
\setluatexcatcodetable \vrbcatcodes {%
  \luatexcatcodetable \CatcodeTableIniTeX
  \catcode  9 \othercatcode %% \tabasciicode
  \catcode 13 \othercatcode %% \endoflineasciicode
  \catcode 12 \othercatcode %% \formfeedasciicode
  \catcode 26 \othercatcode %% \endoffileasciicode
  \catcode 32 \othercatcode %% \spaceasciicode
}

\newluatexcatcodetable \literalcatcodes
\setluatexcatcodetable \literalcatcodes {%
  \luatexcatcodetable \CatcodeTableString
  \catcode 32 \activecatcode %% \spaceasciicode
}

\def \beginlisting {%
  \begingroup
    \luatexcatcodetable \vrbcatcodes
    \beginlistingindeed%
}

\directlua {
  local texprint   = tex.print
  document         = document or { }
  document.printlines = function (buffer)
    for _, line in next, string.explode (buffer, "\noexpand\n") do
      texprint (-1, line)
      texprint (-1, "")
    end
  end
}

\def \beginlistingindeed#1\endlisting{%
  \endgroup
  \begingroup
    \ttfamily
    \small
    \begin {quote}
      \bgroup
        \addfontfeature {RawFeature=-tlig;-liga}%% So one can’t just turn them all off at once using the ``Ligatures`` key?
        \luatexcatcodetable \literalcatcodes
        \obeyspaces
        \obeylines
        \directlua{document.printlines ([==[\detokenize {#1}]==])}
      \egroup
    \end {quote}
  \endgroup
}

%%%%%%%%%%%%%%%%%%%%%%%%%%%%%%%%%%%%%%%%%%%%%%%%%%%%%%%%%%%%%%%%%%%%%%%%%%%%%%%
%% enumerations and lists
%%%%%%%%%%%%%%%%%%%%%%%%%%%%%%%%%%%%%%%%%%%%%%%%%%%%%%%%%%%%%%%%%%%%%%%%%%%%%%%

\def \definelist [#1]#2{% name, itemcode
  \expandafter \def \csname begin#1\endcsname {%
    \begin {itemize}
      \let \normalitem = \item
      \def \altitem ####1{%
        \def \first {####1}%
        #2
      }
      \let \beginnormalitem \item
      \let \endnormalitem   \relax
      \let \beginaltitem    \altitem
      \let \endaltitem      \relax
  }

  \expandafter \def \csname end#1\endcsname {%
    \end {itemize}
  }
}

\definelist  [descriptions]{\normalitem {\textbf \first}\hfill\break}
\definelist   [definitions]{\normalitem {\fileent {\first}}}
\definelist      [filelist]{\normalitem {\fileent {\first}}\space--\hskip 1em}
\definelist  [functionlist]{\normalitem {\luafunction {\first}}\hfill\break}

%%%%%%%%%%%%%%%%%%%%%%%%%%%%%%%%%%%%%%%%%%%%%%%%%%%%%%%%%%%%%%%%%%%%%%%%%%%%%%%
%% columns
%%%%%%%%%%%%%%%%%%%%%%%%%%%%%%%%%%%%%%%%%%%%%%%%%%%%%%%%%%%%%%%%%%%%%%%%%%%%%%%

\def \begindoublecolumns {\begin {multicols} {2}}
\def \enddoublecolumns   {\end {multicols}}

%%%%%%%%%%%%%%%%%%%%%%%%%%%%%%%%%%%%%%%%%%%%%%%%%%%%%%%%%%%%%%%%%%%%%%%%%%%%%%%
%% alignment
%%%%%%%%%%%%%%%%%%%%%%%%%%%%%%%%%%%%%%%%%%%%%%%%%%%%%%%%%%%%%%%%%%%%%%%%%%%%%%%

\def \begincentered {\begin {center}}
\def \endcentered   {\end {center}}

\def \beginnarrower {\begin {quote}}
\def \endnarrower   {\end {quote}}

%%%%%%%%%%%%%%%%%%%%%%%%%%%%%%%%%%%%%%%%%%%%%%%%%%%%%%%%%%%%%%%%%%%%%%%%%%%%%%%
%% special elements
%%%%%%%%%%%%%%%%%%%%%%%%%%%%%%%%%%%%%%%%%%%%%%%%%%%%%%%%%%%%%%%%%%%%%%%%%%%%%%%

\let \beginfrontmatter \relax
\let \endfrontmatter   \relax

\def \beginabstractcontent {\begin {abstract}}
\def \endabstractcontent   {\end {abstract}}

\let \setdocumenttitle      \title
\let \setdocumentdate       \date
\let \setdocumentauthor     \author
\let \typesetdocumenttitle  \maketitle

\AtBeginDocument {%% seriously?
  \let \typesetcontent \tableofcontents%
}

%%%%%%%%%%%%%%%%%%%%%%%%%%%%%%%%%%%%%%%%%%%%%%%%%%%%%%%%%%%%%%%%%%%%%%%%%%%%%%%
%% floats
%%%%%%%%%%%%%%%%%%%%%%%%%%%%%%%%%%%%%%%%%%%%%%%%%%%%%%%%%%%%%%%%%%%%%%%%%%%%%%%

%% syntax definition
\def \beginsyntaxfloat #1#2{%% #1:label #2:caption
  \begin {figure} [b]
    \edef \syntaxlabel  {#1}%
    \def \syntaxcaption {#2}%
    \setlength\grammarparsep{12pt plus 2pt minus 2pt}%
    \setlength\grammarindent{5cm}%
    \begingroup
      \small
      \begin {grammar}
}

\def \endsyntaxfloat {%
      \end {grammar}
    \endgroup
    \caption \syntaxcaption
    \label   \syntaxlabel
  \end {figure}
}

%% figures, e.g. the file graph
\def \figurefloat #1#2#3{%% #1:label #2:caption #3:file
  \begin {figure} [b]
    \caption {#2}%
    \includegraphics[width=\textwidth]{#3}%
    \label {#1}
  \end {figure}
}

%% tables
\def \tablefloat #1#2{%% #1:label #2:caption
  \begin {table} [t]
    \hrule
    \caption {#2}%
    \label {#1}
    \hrule
  \end {table}
}


%%%%%%%%%%%%%%%%%%%%%%%%%%%%%%%%%%%%%%%%%%%%%%%%%%%%%%%%%%%%%%%%%%%%%%%%%%%%%%%
%% hyperlinks
%%%%%%%%%%%%%%%%%%%%%%%%%%%%%%%%%%%%%%%%%%%%%%%%%%%%%%%%%%%%%%%%%%%%%%%%%%%%%%%

\protected \def \hyperlink{%
  \@ifnextchar[{\hyperlinkindeed}%
               {\hyperlinkindeed[]}%
}

\def \hyperlinkindeed [#1]#2{%
  \def \first {#1}%
  \ifx \first \empty
    \url {#2}%
  \else
    \href {#2}{#1}%
  \fi%
}


%%%%%%%%%%%%%%%%%%%%%%%%%%%%%%%%%%%%%%%%%%%%%%%%%%%%%%%%%%%%%%%%%%%%%%%%%%%%%%%
%% tables
%%%%%%%%%%%%%%%%%%%%%%%%%%%%%%%%%%%%%%%%%%%%%%%%%%%%%%%%%%%%%%%%%%%%%%%%%%%%%%%
%% Our tables aren’t anything special so we stick with “tabular” on the
%% Latex end.
%%
%% This is going to be largely incompatible with Context since format
%% specifications work quite differently (even between different
%% Context table variants).

\def \begintabulate [#1]#2\endtabulate{%
  \begingroup
    \let \beginrow  = \relax %% -> \NC in Context
    \let \newcell   = &      %% -> \NC
    \let \endrow    = \cr    %% -> \NC \NR
    \begin {tabular}{#1}%
      #2
    \end {tabular}
  \endgroup
}

\let \endtabulate \relax

%%%%%%%%%%%%%%%%%%%%%%%%%%%%%%%%%%%%%%%%%%%%%%%%%%%%%%%%%%%%%%%%%%%%%%%%%%%%%%%
%% escaped characters
%%%%%%%%%%%%%%%%%%%%%%%%%%%%%%%%%%%%%%%%%%%%%%%%%%%%%%%%%%%%%%%%%%%%%%%%%%%%%%%

\let \charpercent   \textpercent
\let \charbackslash \textbackslash
\let \chartilde     \textasciitilde

%%%%%%%%%%%%%%%%%%%%%%%%%%%%%%%%%%%%%%%%%%%%%%%%%%%%%%%%%%%%%%%%%%%%%%%%%%%%%%%
%% main
%%%%%%%%%%%%%%%%%%%%%%%%%%%%%%%%%%%%%%%%%%%%%%%%%%%%%%%%%%%%%%%%%%%%%%%%%%%%%%%

\makeatother

\newif \ifcontextmkiv \contextmkivfalse

\begin {document}
  %% Copyright (C) 2009-2014
%%
%%      by  Elie Roux      <elie.roux@telecom-bretagne.eu>
%%      and Khaled Hosny   <khaledhosny@eglug.org>
%%      and Philipp Gesang <philipp.gesang@alumni.uni-heidelberg.de>
%%
%% This file is part of Luaotfload.
%%
%%      Home:      https://github.com/lualatex/luaotfload
%%      Support:   <lualatex-dev@tug.org>.
%%
%% Luaotfload is under the GPL v2.0 (exactly) license.
%%
%% ----------------------------------------------------------------------------
%%
%% Luaotfload is free software; you can redistribute it and/or
%% modify it under the terms of the GNU General Public License
%% as published by the Free Software Foundation; version 2
%% of the License.
%%
%% Luaotfload is distributed in the hope that it will be useful,
%% but WITHOUT ANY WARRANTY; without even the implied warranty of
%% MERCHANTABILITY or FITNESS FOR A PARTICULAR PURPOSE. See the
%% GNU General Public License for more details.
%%
%% You should have received a copy of the GNU General Public License
%% along with Luaotfload; if not, see <http://www.gnu.org/licenses/>.
%%
%% ----------------------------------------------------------------------------
%%

\setdocumenttitle  {The \identifier{luaotfload} package}
\setdocumentdate   {2014/**/** v2.5}
\setdocumentauthor {Elie Roux · Khaled Hosny · Philipp Gesang\\
                    Home:    \hyperlink {https://github.com/lualatex/luaotfload}\\
                    Support: \email     {lualatex-dev@tug.org}}

\typesetdocumenttitle

\beginabstractcontent
  This package is an adaptation of the \CONTEXT font loading system.
  It allows for loading \OpenType fonts with an extended syntax and adds
  support for a variety of font features.
\endabstractcontent

\tableofcontents

%%%%%%%%%%%%%%%%%%%%%%%%%%%%%%%%%%%%%%%%%%%%%%%%%%%%%%%%%%%%%%%%%%%%%%%%%%%%%%%
\beginsection {Introduction}
%%%%%%%%%%%%%%%%%%%%%%%%%%%%%%%%%%%%%%%%%%%%%%%%%%%%%%%%%%%%%%%%%%%%%%%%%%%%%%%

Font management and installation has always been painful with \TEX.  A
lot of files are needed for one font (\abbrev{tfm}, \abbrev{pfb},
\abbrev{map}, \abbrev{fd}, \abbrev{vf}), and due to the 8-Bit encoding
each font is limited to 256 characters.

But the font world has evolved since the original \TEX, and new
typographic systems have appeared, most notably the so called
\emphasis{smart font} technologies like \OpenType fonts (\abbrev{otf}).

These fonts can contain many more characters than \TEX fonts, as well
as additional functionality like ligatures, old-style numbers, small
capitals, etc., and support more complex writing systems like Arabic
and Indic\footnote{%
  Unfortunately, \identifier{luaotfload} doesn‘t support many Indic
  scripts right now.
  Assistance in implementing the prerequisites is greatly
  appreciated.
}
scripts.

\OpenType fonts are widely deployed and available for all modern
operating systems.

As of 2013 they have become the de facto standard for advanced text
layout.

However, until recently the only way to use them directly in the \TEX
world was with the \XETEX engine.

Unlike \XETEX, \LUATEX has no built-in support for \OpenType or
technologies other than the original \TEX fonts.

Instead, it provides hooks for executing \LUA code during the \TEX run
that allow implementing extensions for loading fonts and manipulating
how input text is processed without modifying the underlying engine.

This is where \identifier{luaotfload} comes into play:
Based on code from \CONTEXT, it extends \LUATEX with functionality necessary
for handling \OpenType fonts.

Additionally, it provides means for accessing fonts known to the operating
system conveniently by indexing the metadata.

\endsection

%%%%%%%%%%%%%%%%%%%%%%%%%%%%%%%%%%%%%%%%%%%%%%%%%%%%%%%%%%%%%%%%%%%%%%%%%%%%%%%
\beginsection {Thanks}
%%%%%%%%%%%%%%%%%%%%%%%%%%%%%%%%%%%%%%%%%%%%%%%%%%%%%%%%%%%%%%%%%%%%%%%%%%%%%%%

\identifier{Luaotfload} is part of \LUALATEX, the community-driven
project to provide a foundation for using the \LATEX format with the
full capabilites of the \LUATEX engine.
%
As such, the distinction between end users, contributors, and project
maintainers is intentionally kept less strict, lest we unduly
personalize the common effort.

Nevertheless, the current maintainers would like to express their
gratitude to Khaled Hosny, Akira Kakuto, Hironori Kitagawa and Dohyun
Kim.
%
Their contributions -- be it patches, advice, or systematic
testing -- made the switch from version 1.x to 2.2 possible.
%
Also, Hans Hagen, the author of the font loader, made porting the
code to \LATEX a breeze due to the extra effort he invested into
isolating it from the rest of \CONTEXT, not to mention his assistance
in the task and willingness to respond to our suggestions.

\endsection

%%%%%%%%%%%%%%%%%%%%%%%%%%%%%%%%%%%%%%%%%%%%%%%%%%%%%%%%%%%%%%%%%%%%%%%%%%%%%%%
\beginsection {Loading Fonts}
%%%%%%%%%%%%%%%%%%%%%%%%%%%%%%%%%%%%%%%%%%%%%%%%%%%%%%%%%%%%%%%%%%%%%%%%%%%%%%%

\identifier{luaotfload} supports an extended font request syntax:

\beginnarrower
      |\font\foo={|%
      \meta{prefix}|:|%
      \meta{font name}|:|%
      \meta{font features}|}|%
      \meta{\TEX font features}
\endnarrower

\noindent
The curly brackets are optional and escape the spaces in the enclosed
font name.
%
Alternatively, double quotes serve the same purpose.
%
A selection of individual parts of the syntax are discussed below;
for a more formal description see figure \ref{font-syntax}.

\beginsyntaxfloat
  {font-syntax}
  {Font request syntax.
   Braces or double quotes around the
   \emphasis{specification} rule will
   preserve whitespace in file names.
   In addition to the font style modifiers
   (\emphasis{slash-notation}) given above, there
   are others that are recognized but will be silently
   ignored: {\ttfamily aat},
            {\ttfamily icu}, and
            {\ttfamily gr}.
   The special terminals are:
   {\sc feature\textunderscore id} for a valid font
      feature name and
   {\sc feature\textunderscore value} for the corresponding
      value.
   {\sc tfmname} is the name of a \abbrev{tfm} file.
   {\sc digit}  again refers to bytes 48--57, and
   {\sc all\textunderscore characters} to all byte values.
   {\sc csname} and {\sc dimension} are the \TEX concepts.}
%
      <definition>      ::= `\\font', {\sc csname}, `=', <font request>, [ <size> ] ;

      <size>            ::= `at', {\sc dimension} ;

      <font request>    ::= `"', <unquoted font request> `"'
      \alt                  `{', <unquoted font request> `}'
      \alt                  <unquoted font request> ;

      <unquoted font request> ::= <specification>, [`:', <feature list> ]
      \alt                        `[', <path lookup> `]', [ [`:'], <feature list> ] ;

      <specification>    ::= <prefixed spec>, [ <subfont no> ], \{ <modifier> \}
      \alt                   <anon lookup>, \{ <modifier> \} ;

      <prefixed spec>    ::= `file:', <file lookup>
      \alt                   `name:', <name lookup> ;

      <file lookup>      ::= \{ <name character> \} ;

      <name lookup>      ::= \{ <name character> \} ;

      <anon lookup>      ::= {\sc tfmname} | <name lookup> ;

      <path lookup>      ::= \{ {\sc all_characters} - `]' \} ;

      <modifier>         ::= `/', (`I' | `B' | `BI' | `IB' | `S=', \{ {\sc digit} \} ) ;

      <subfont no>       ::= `(', \{ {\sc digit} \}, `)' ;

      <feature list>     ::= <feature expr>, \{ `;', <feature expr> \} ;

      <feature expr>     ::= {\sc feature_id}, `=', {\sc feature_value}
      \alt                   <feature switch>, {\sc feature_id} ;

      <feature switch>   ::= `+' | `-' ;

      <name character>   ::= {\sc all_characters} - ( `(' | `/' | `:' ) ;
\endsyntaxfloat

\beginsubsection{Prefix -- the \identifier{luaotfload}{ }Way}

In \identifier{luaotfload}, the canonical syntax for font requests
requires a \emphasis{prefix}:
%
\beginnarrower
  |\font\fontname=|\meta{prefix}|:|\meta{fontname}\dots
\endnarrower
%
where \meta{prefix} is either \inlinecode{file:} or \inlinecode {name:}.\footnote{%
  The development version also knows two further prefixes,
  \inlinecode {kpse:} and \inlinecode {my:}.
  %
  A \inlinecode {kpse} lookup is restricted to files that can be found by
  \identifier{kpathsea} and
  will not attempt to locate system fonts.
  %
  This behavior can be of value when an extra degree of encapsulation is
  needed, for instance when supplying a customized tex distribution.

  The \inlinecode {my} lookup takes this a step further: it lets you define
  a custom resolver function and hook it into the \luafunction{resolve_font}
  callback.
  %
  This ensures full control over how a file is located.
  %
  For a working example see the
  \hyperlink [test repo]{https://bitbucket.org/phg/lua-la-tex-tests/src/5f6a535d/pln-lookup-callback-1.tex}.
}
%
It determines whether the font loader should interpret the request as
a \emphasis{file name} or
  \emphasis{font name}, respectively,
which again influences how it will attempt to locate the font.
%
Examples for font names are
            “Latin Modern Italic”,
            “GFS Bodoni Rg”, and
            “PT Serif Caption”
-- they are the human readable identifiers
usually listed in drop-down menus and the like.\footnote{%
  Font names may appear like a great choice at first because they
  offer seemingly more intuitive identifiers in comparison to arguably
  cryptic file names:
  %
  “PT Sans Bold” is a lot more descriptive than \fileent{PTS75F.ttf}.
  On the other hand, font names are quite arbitrary and there is no
  universal method to determine their meaning.
  %
  While \identifier{luaotfload} provides fairly sophisticated heuristic
  to figure out a matching font style, weight, and optical size, it
  cannot be relied upon to work satisfactorily for all font files.
  %
  For an in-depth analysis of the situation and how broken font names
  are, please refer to
  \hyperlink [this post]{http://www.ntg.nl/pipermail/ntg-context/2013/073889.html}
  by Hans Hagen, the author of the font loader.
  %
  If in doubt, use filenames.
  %
  \fileent{luaotfload-tool} can perform the matching for you with the
  option \inlinecode {--find=<name>}, and you can use the file name it returns
  in your font definition.
}
%
In order for fonts installed both in system locations and in your
\fileent{texmf} to be accessible by font name, \identifier{luaotfload} must
first collect the metadata included in the files.
%
Please refer to section~\ref{sec:fontdb} below for instructions on how to
create the database.

File names are whatever your file system allows them to be, except
that that they may not contain the characters
  \inlinecode {(},
  \inlinecode {:}, and
  \inlinecode {/}.
%
As is obvious from the last exception, the \inlinecode {file:} lookup will
not process paths to the font location -- only those
files found when generating the database are addressable this way.
%
Continue below in the \XETEX section if you need to load your fonts
by path.
%
The file names corresponding to the example font names above are
  \fileent{lmroman12-italic.otf},
  \fileent{GFSBodoni.otf}, and
  \fileent{PTZ56F.ttf}.

\endsubsection

\beginsubsection {Compatibility Layer}

In addition to the regular prefixed requests, \identifier{luaotfload}
accepts loading fonts the \XETEX way.
%
There are again two modes: bracketed and unbracketed.
A bracketed request looks as follows.

\beginnarrower
  |\font\fontname=[|\meta{path to file}|]|
\endnarrower

\noindent
Inside the square brackets, every character except for a closing
bracket is permitted, allowing for specifying paths to a font file.
%
Naturally, path-less file names are equally valid and processed the
same way as an ordinary \inlinecode {file:} lookup.

\beginnarrower
  |\font\fontname=|\meta{font name} \dots
\endnarrower

Unbracketed (or, for lack of a better word: \emphasis{anonymous})
font requests resemble the conventional \TEX syntax.
%
However, they have a broader spectrum of possible interpretations:
before anything else, \identifier{luaotfload} attempts to load a
traditional \TEX Font Metric (\abbrev{tfm} or \abbrev{ofm}).
%
If this fails, it performs a \inlinecode {name:} lookup, which itself will
fall back to a \inlinecode {file:} lookup if no database entry matches
\meta{font name}.

Furthermore, \identifier{luaotfload} supports the slashed (shorthand)
font style notation from \XETEX.

\beginnarrower
  |\font\fontname=|\meta{font name}|/|\meta{modifier}\dots
\endnarrower

\noindent
Currently, four style modifiers are supported:
  \inlinecode {I} for italic shape,
  \inlinecode {B} for bold   weight,
  \inlinecode {BI} or \inlinecode {IB} for the combination of both.
%
Other “slashed” modifiers are too specific to the \XETEX engine and
have no meaning in \LUATEX.

\endsubsection

\beginsubsection{Examples}

\beginsubsubsection{Loading by File Name}

For example, conventional \abbrev{type1} font can be loaded with a
\inlinecode {file:} request like so:

\beginlisting
  \font \lmromanten = {file:ec-lmr10} at 10pt
\endlisting

The \OpenType version of Janusz Nowacki’s font \emphasis{Antykwa
Półtawskiego}\footnote{%
  \hyperlink {http://jmn.pl/antykwa-poltawskiego/}, also available in
  in \TEX Live.
}
in its condensed variant can be loaded as follows:

\beginlisting
  \font \apcregular = file:antpoltltcond-regular.otf at 42pt
\endlisting

The next example shows how to load the \emphasis{Porson} font digitized by
the Greek Font Society using \XETEX-style syntax and an absolute path from a
non-standard directory:

\beginlisting
  \font \gfsporson = "[/tmp/GFSPorson.otf]" at 12pt
\endlisting

\endsubsubsection

\beginsubsubsection{Loading by Font Name}

The \inlinecode {name:} lookup does not depend on cryptic filenames:

\beginlisting
  \font \pagellaregular = {name:TeX Gyre Pagella} at 9pt
\endlisting

A bit more specific but essentially the same lookup would be:

\beginlisting
  \font \pagellaregular = {name:TeX Gyre Pagella Regular} at 9pt
\endlisting

\noindent
Which fits nicely with the whole set:

\beginlisting
  \font\pagellaregular    = {name:TeX Gyre Pagella Regular}    at 9pt
  \font\pagellaitalic     = {name:TeX Gyre Pagella Italic}     at 9pt
  \font\pagellabold       = {name:TeX Gyre Pagella Bold}       at 9pt
  \font\pagellabolditalic = {name:TeX Gyre Pagella Bolditalic} at 9pt

  {\pagellaregular     foo bar baz\endgraf}
  {\pagellaitalic      foo bar baz\endgraf}
  {\pagellabold        foo bar baz\endgraf}
  {\pagellabolditalic  foo bar baz\endgraf}

  ...
\endlisting

\endsubsubsection

\beginsubsubsection{Modifiers}

If the entire \emphasis{Iwona} family\footnote{%
  \hyperlink {http://jmn.pl/kurier-i-iwona/},
  also in \TEX Live.
}
is installed in some location accessible by \identifier{luaotfload},
the regular shape can be loaded as follows:

\beginlisting
  \font\iwona=Iwona at 20pt
\endlisting

\noindent
To load the most common of the other styles, the slash notation can
be employed as shorthand:

\beginlisting
  \font\iwonaitalic    =Iwona/I    at 20pt
  \font\iwonabold      =Iwona/B    at 20pt
  \font\iwonabolditalic=Iwona/BI   at 20pt
\endlisting

\noindent
which is equivalent to these full names:

\beginlisting
  \font\iwonaitalic    ="Iwona Italic"       at 20pt
  \font\iwonabold      ="Iwona Bold"         at 20pt
  \font\iwonabolditalic="Iwona BoldItalic"   at 20pt
\endlisting

\endsubsubsection
\endsubsection
\endsection

%%%%%%%%%%%%%%%%%%%%%%%%%%%%%%%%%%%%%%%%%%%%%%%%%%%%%%%%%%%%%%%%%%%%%%%%%%%%%%%
\beginsection {Font features}
%%%%%%%%%%%%%%%%%%%%%%%%%%%%%%%%%%%%%%%%%%%%%%%%%%%%%%%%%%%%%%%%%%%%%%%%%%%%%%%

\emphasis{Font features} are the second to last component in the
general scheme for font requests:

\beginnarrower
      |\font\foo={|%
      \meta{prefix}|:|%
      \meta{font name}|:|%
      \meta{font features}|}|%
      \meta{\TEX font features}
\endnarrower

\noindent
If style modifiers are present (\XETEX style), they must precede
\meta{font features}.

The element \meta{font features} is a semicolon-separated list of feature
tags\footnote{%
  Cf. \hyperlink {http://www.microsoft.com/typography/otspec/featurelist.htm}.
}
and font options.
%
Prepending a font feature with a |+| (plus sign) enables it, whereas
a |-| (minus) disables it. For instance, the request

\beginlisting
  \font\test=LatinModernRoman:+clig;-kern
\endlisting

\noindent activates contextual ligatures (|clig|) and disables
kerning (|kern|).
%
Alternatively the options |true| or |false| can be passed to
the feature in a key/value expression.
%
The following request has the same meaning as the last one:

\beginlisting
  \font\test=LatinModernRoman:clig=true;kern=false
\endlisting

\noindent
Furthermore, this second syntax is required should a font feature
accept other options besides a true/false switch.
%
For example, \emphasis{stylistic alternates} (|salt|) are variants of
given glyphs.
%
They can be selected either explicitly by supplying the variant
index (starting from one), or randomly by setting the value to,
obviously, |random|.

%% TODO   verify that this actually works with a font that supports
%%        the salt/random feature!\fi
\beginlisting
  \font\librmsaltfirst=LatinModernRoman:salt=1
\endlisting

\beginsubsection {Basic font features}

\begindescriptions

  \altitem {mode}
         \identifier{luaotfload} has two \OpenType processing
         \emphasis{modes}:
         \identifier{base} and \identifier{node}.

         \identifier{base} mode works by mapping \OpenType
         features to traditional \TEX ligature and kerning mechanisms.
         %
         Supporting only non-contextual substitutions and kerning
         pairs, it is the slightly faster, albeit somewhat limited, variant.
         %
         \identifier{node} mode works by processing \TeX’s internal
         node list directly at the \LUA end and supports
         a wider range of \OpenType features.
         %
         The downside is that the intricate operations required for
         \identifier{node} mode may slow down typesetting especially
         with complex fonts and it does not work in math mode.

         By default \identifier{luaotfload} is in \identifier{node}
         mode, and \identifier{base} mode has to be requested where needed,
         e.~g. for math fonts.

  \altitem {script} \label{script-tag}
         An \OpenType script tag;\footnote{%
           See \hyperlink {http://www.microsoft.com/typography/otspec/scripttags.htm}
           for a list of valid values.
           %
           For scripts derived from the Latin alphabet the value
           |latn| is good choice.
         }
         the default value is |dlft|.
         %
         Some fonts, including very popular ones by foundries like Adobe,
         do not assign features to the |dflt| script, in
         which case the script needs to be set explicitly.

  \altitem {language}
         An \OpenType language system identifier,\footnote{%
           Cf. \hyperlink {http://www.microsoft.com/typography/otspec/languagetags.htm}.
         }
         defaulting to |dflt|.

  \altitem {featurefile}
         A comma-separated list of feature files to be applied to the
         font.
         %
         Feature files contain a textual representation of
         \OpenType tables and extend the features of a font
         on fly.
         %
         After they are applied to a font, features defined in a
         feature file can be enabled or disabled just like any
         other font feature.
         %
         The syntax is documented in \identifier{Adobe}’s
         \OpenType Feature File Specification.\footnote{%
           Cf. \hyperlink {http://www.adobe.com/devnet/opentype/afdko/topic_feature_file_syntax.html}.
           Feature file support is part of the engine which at the
           time of this writing (2014) implements the spec only
           partially.
           See the
           \hyperlink [\LUATEX tracker]{http://tracker.luatex.org/view.php?id=231}
            for details.
         }

         For a demonstration of how to set a |tkrn| feature consult
         the file |tkrn.fea| that is part of \identifier{luaotfload}.
         It can be read and applied as follows:

         |\font\test=Latin Modern Roman:featurefile=tkrn.fea;+tkrn|

  \altitem {color}
         A font color, defined as a triplet of two-digit hexadecimal
         \abbrev{rgb} values, with an optional fourth value for
         transparency
         (where |00| is completely transparent and |FF| is opaque).

         For example, in order to set text in semitransparent red:

         \beginlisting
\font\test={Latin Modern Roman}:color=FF0000BB
         \endlisting

  \altitem {kernfactor \& letterspace}
         Define a font with letterspacing (tracking) enabled.
         %
         In \identifier{luaotfload}, letterspacing is implemented by
         inserting additional kerning between glyphs.

         This approach is derived from and still quite similar to the
         \emphasis{character kerning} (\texmacro{setcharacterkerning} /
         \texmacro{definecharacterkerning} \& al.) functionality of
         Context, see the file \fileent{typo-krn.lua} there.
         %
         The main difference is that \identifier{luaotfload} does not
         use \LUATEX attributes to assign letterspacing to regions,
         but defines virtual letterspaced versions of a font.

         The option \identifier{kernfactor} accepts a numeric value that
         determines the letterspacing factor to be applied to the font
         size.
         %
         E.~g. a kern factor of $0.42$ applied to a $10$ pt font
         results in $4.2$ pt of additional kerning applied to each
         pair of glyphs.
         %
         Ligatures are split into their component glyphs unless
         explicitly ignored (see below).

         For compatibility with \XETEX an alternative
         \identifier{letterspace} option is supplied that interprets the
         supplied value as a \emphasis{percentage} of the font size but
         is otherwise identical to \identifier{kernfactor}.
         %
         Consequently, both definitions in below snippet yield the same
         letterspacing width:

         \beginlisting
\font\iwonakernedA="file:Iwona-Regular.otf:kernfactor=0.125"
\font\iwonakernedB="file:Iwona-Regular.otf:letterspace=12.5"
         \endlisting

         Specific pairs of letters and ligatures may be exempt from
         letterspacing by defining the \LUA functions
         \luafunction{keeptogether} and \luafunction{keepligature},
         respectively, inside the namespace \inlinecode {luaotfload.letterspace}.
         %
         Both functions are called whenever the letterspacing callback
         encounters an appropriate node or set of nodes.
         %
         If they return a true-ish value, no extra kern is inserted at
         the current position.
         %
         \luafunction{keeptogether} receives a pair of consecutive
         glyph nodes in order of their appearance in the node list.
         %
         \luafunction{keepligature} receives a single node which can be
         analyzed into components.
         %
         (For details refer to the \emphasis{glyph nodes} section in the
         \LUATEX reference manual.)
         %
         The implementation of both functions is left entirely to the
         user.


  \altitem {protrusion \& expansion}
         These keys control microtypographic features of the font,
         namely \emphasis{character protrusion} and \emphasis{font
         expansion}.
         %
         Their arguments are names of \LUA tables that contain
         values for the respective features.\footnote{%
            For examples of the table layout please refer to the
            section of the file \fileent{luaotfload-fonts-ext.lua} where the
            default values are defined.
            %
            Alternatively and with loss of information, you can dump
            those tables into your terminal by issuing
            \beginlisting
\directlua{inspect(fonts.protrusions.setups.default)
           inspect(fonts.expansions.setups.default)}
            \endlisting
            at some point after loading \fileent{luaotfload.sty}.
         }
         %
         For both, only the set \identifier{default} is predefined.

         For example, to define a font with the default
         protrusion vector applied\footnote{%
           You also need to set
               \inlinecode {pdfprotrudechars=2} and
               \inlinecode {pdfadjustspacing=2}
           to activate protrusion and expansion, respectively.
           See the
           \hyperlink [\PDFTEX manual]{http://mirrors.ctan.org/systems/pdftex/manual/pdftex-a.pdf}%
           for details.
         }:

         \beginlisting
\font\test=LatinModernRoman:protrusion=default
         \endlisting
\enddescriptions

\endsubsection

\beginsubsection {Non-standard font features}
\identifier{luaotfload} adds a number of features that are not defined
in the original \OpenType specification, most of them
aiming at emulating the behavior familiar from other \TEX engines.
%
Currently (2014) there are three of them:

\begindescriptions

  \altitem {anum}
          Substitutes the glyphs in the \abbrev{ascii} number range
          with their counterparts from eastern Arabic or Persian,
          depending on the value of \identifier{language}.

  \altitem {tlig}
          Applies legacy \TEX ligatures:

          \begin{tabular}{rlrl}
             ``  &  \inlinecode {``}  &  ''  &  \inlinecode {''}  \\
             `   &  \inlinecode {`}   &  '   &  \inlinecode {'}   \\
             "   &  \inlinecode {"}   &  --  &  \inlinecode {--}  \\
             --- &  \inlinecode {---} &  !`  &  \inlinecode {!`}  \\
             ?`  &  \inlinecode {?`}  &      &                    \\
          \end{tabular}

          \footnote{%
            These contain the feature set \inlinecode {trep} of earlier
            versions of \identifier{luaotfload}.

            Note to \XETEX users: this is the equivalent of the
            assignment \inlinecode {mapping=text-tex} using \XETEX's input
            remapping feature.
          }

  \altitem {itlc}
          Computes italic correction values (active by default).

\enddescriptions

\endsubsection
\endsection

%%%%%%%%%%%%%%%%%%%%%%%%%%%%%%%%%%%%%%%%%%%%%%%%%%%%%%%%%%%%%%%%%%%%%%%%%%%%%%%
\beginsection {Font names database}
%%%%%%%%%%%%%%%%%%%%%%%%%%%%%%%%%%%%%%%%%%%%%%%%%%%%%%%%%%%%%%%%%%%%%%%%%%%%%%%

\label{sec:fontdb}

As mentioned above, \identifier{luaotfload} keeps track of which
fonts are available to \LUATEX by means of a \emphasis{database}.
%
This allows referring to fonts not only by explicit filenames but
also by the proper names contained in the metadata which is often
more accessible to humans.\footnote{%
  The tool \hyperlink[\fileent{otfinfo}]{http://www.lcdf.org/type/}
  (comes with \TEX Live), when invoked on a font file with the
  \inlinecode {-i} option, lists the variety of name fields defined for
  it.
}

When \identifier{luaotfload} is asked to load a font by a font name,
it will check if the database exists and load it, or else generate a
fresh one.
%
Should it then fail to locate the font, an update to the database is
performed in case the font has been added to the system only
recently.
%
As soon as the database is updated, the resolver will try
and look up the font again, all without user intervention.
%
The goal is for \identifier{luaotfload} to act in the background and
behave as unobtrusively as possible, while providing a convenient
interface to the fonts installed on the system.

Generating the database for the first time may take a while since it
inspects every font file on your computer.
%
This is particularly noticeable if it occurs during a typesetting run.
In any case, subsequent updates to the database will be quite fast.

\beginsubsection[luaotfload-tool]
                {\fileent{luaotfload-tool}}

It can still be desirable at times to do some of these steps
manually, and without having to compile a document.
%
To this end, \identifier{luaotfload} comes with the utility
\fileent{luaotfload-tool} that offers an interface to the database
functionality.
%
Being a \LUA script, there are two ways to run it:
either make it executable (\inlinecode {chmod +x} on unixoid systems) or
pass it as an argument to \fileent{texlua}.\footnote{%
  Tests by the maintainer show only marginal performance gain by
  running with Luigi Scarso’s
  \hyperlink [\identifier{Luajit\kern-.25ex\TEX}]{https://foundry.supelec.fr/projects/luajittex/},
  which is probably due to the fact that most of the time is spent
  on file system operations.

  \emphasis{Note}:
  On \abbrev{MS} \identifier{Windows} systems, the script can be run
  either by calling the wrapper application
  \fileent{luaotfload-tool.exe} or as
  \inlinecode {texlua.exe luaotfload-tool.lua}.
}
%
Invoked with the argument \inlinecode {--update} it will perform a database
update, scanning for fonts not indexed.

\beginlisting
  luaotfload-tool --update
\endlisting

Adding the \inlinecode {--force} switch will initiate a complete
rebuild of the database.

\beginlisting
  luaotfload-tool --update --force
\endlisting
Whenever it is run under this name, it will update the database
first, mimicking the behavior of earlier versions of
\identifier{luaotfload}.

\endsubsection

\beginsubsection{Search Paths}

\identifier{luaotfload} scans those directories where fonts are
expected to be located on a given system.
%
On a Linux machine it follows the paths listed in the
\identifier{Fontconfig} configuration files;
consult \inlinecode {man 5 fonts.conf} for further information.
%
On \identifier{Windows} systems, the standard location is
\inlinecode {Windows\\Fonts},
%
while \identifier{Mac OS~X} requires a multitude of paths to
be examined.
%
The complete list is is given in table \ref{table-searchpaths}.
Other paths can be specified by setting the environment variable
\inlinecode {OSFONTDIR}.
%
If it is non-empty, then search will be extended to the included
directories.

\begin{table}[t]
  \hrule
  \caption{List of paths searched for each supported operating
           system.}
  \renewcommand{\arraystretch}{1.2}
  \begincentered
    \begin{tabular}{lp{.5\textwidth}}
      Windows     & \inlinecode {\%WINDIR\%\\Fonts}
      \\
      Linux       & \fileent{/usr/local/etc/fonts/fonts.conf} and\hfill\break
                    \fileent{/etc/fonts/fonts.conf}
      \\
      Mac         & \fileent{\textasciitilde/Library/Fonts},\break
                    \fileent{/Library/Fonts},\break
                    \fileent{/System/Library/Fonts}, and\hfill\break
                    \fileent{/Network/Library/Fonts}
      \\
    \end{tabular}
  \endcentered
  \label{table-searchpaths}
  \hrule
\end{table}

\endsubsection

\beginsubsection{Querying from Outside}

\fileent{luaotfload-tool} also provides rudimentary means of
accessing the information collected in the font database.
%
If the option \inlinecode {--find=}\emphasis{name} is given, the script will
try and search the fonts indexed by \identifier{luaotfload} for a
matching name.
%
For instance, the invocation

\beginlisting
  luaotfload-tool  --find="Iwona Regular"
\endlisting

\noindent
will verify if “Iwona Regular” is found in the database and can be
readily requested in a document.

If you are unsure about the actual font name, then add the
\inlinecode {-F} (or \inlinecode {--fuzzy}) switch to the command line to enable
approximate matching.
%
Suppose you cannot precisely remember if the variant of
\identifier{Iwona} you are looking for was “Bright” or “Light”.
The query

\beginlisting
  luaotfload-tool  -F --find="Iwona Bright"
\endlisting

\noindent
will tell you that indeed the latter name is correct.

Basic information about fonts in the database can be displayed
using the \inlinecode {-i} option (\inlinecode {--info}).
%
\beginlisting
  luaotfload-tool  -i --find="Iwona Light Italic"
\endlisting
%
\noindent
The meaning of the printed values is described in section 4.4 of the
\LUATEX reference manual.\footnote{%
  In \TEX Live: \fileent{texmf-dist/doc/luatex/base/luatexref-t.pdf}.
}

For a much more detailed report about a given font try the
\inlinecode {-I} option instead (\inlinecode {--inspect}).
\beginlisting
  luaotfload-tool  -I --find="Iwona Light Italic"
\endlisting

\inlinecode {luaotfload-tool --help} will list the available command line
switches, including some not discussed in detail here.
%
For a full documentation of \identifier{luaotfload-tool} and its
capabilities refer to the manpage
(\inlinecode {man 1 luaotfload-tool}).\footnote{%
  Or see \inlinecode {luaotfload-tool.rst} in the source directory.
}

\endsubsection

\beginsubsection {Blacklisting Fonts}
\label{font-blacklist}

Some fonts are problematic in general, or just in \LUATEX.
%
If you find that compiling your document takes far too long or eats
away all your system’s memory, you can track down the culprit by
running \inlinecode {luaotfload-tool -v} to increase verbosity.
%
Take a note of the \emphasis{filename} of the font that database
creation fails with and append it to the file
\fileent{luaotfload-blacklist.cnf}.

A blacklist file is a list of font filenames, one per line.
Specifying the full path to where the file is located is optional, the
plain filename should suffice.
%
File extensions (\fileent{.otf}, \fileent{.ttf}, etc.) may be omitted.
%
Anything after a percent (|%|) character until the end of the line
is ignored, so use this to add comments.
%
Place this file to some location where the \identifier{kpse}
library can find it, e.~g.
\fileent{texmf-local/tex/luatex/luaotfload} if you are running
\identifier{\TEX Live},\footnote{%
  You may have to run \inlinecode {mktexlsr} if you created a new file in
  your \fileent{texmf} tree.
}
or just leave it in the working directory of your document.
%
\identifier{luaotfload} reads all files named
\fileent{luaotfload-blacklist.cnf} it finds, so the fonts in
\fileent{./luaotfload-blacklist.cnf} extend the global blacklist.

Furthermore, a filename prepended with a dash character (|-|) is
removed from the blacklist, causing it to be temporarily whitelisted
without modifying the global file.
%
An example with explicit paths:

\beginlisting
% example otf-blacklist.cnf
/Library/Fonts/GillSans.ttc  % Luaotfload ignores this font.
-/Library/Fonts/Optima.ttc   % This one is usable again, even if
                             % blacklisted somewhere else.
\endlisting

\endsubsection
\endsection

%%%%%%%%%%%%%%%%%%%%%%%%%%%%%%%%%%%%%%%%%%%%%%%%%%%%%%%%%%%%%%%%%%%%%%%%%%%%%%%
\beginsection {Files from \CONTEXT and \LUATEX-Fonts}
%%%%%%%%%%%%%%%%%%%%%%%%%%%%%%%%%%%%%%%%%%%%%%%%%%%%%%%%%%%%%%%%%%%%%%%%%%%%%%%

\identifier{luaotfload} relies on code originally written by Hans
Hagen for the \hyperlink[\identifier\CONTEXT]{http://wiki.contextgarden.net}
format.
%
It integrates the font loader as distributed in
the \identifier{\LUATEX-Fonts} package.
%
The original \LUA source files have been combined using the
\fileent{mtx-package} script into a single, self-contained blob.
In this form the font loader has no further dependencies\footnote{%
  It covers, however, to some extent the functionality of the
  \identifier{lualibs} package.
}
and requires only minor adaptions to integrate into
\identifier{luaotfload}.
%
The guiding principle is to let \CONTEXT/\LUATEX-Fonts take care of
the implementation, and update the imported code from time to time.
%
As maintainers, we aim at importing files from upstream essentially
\emphasis{unmodified}, except for renaming them to prevent name
clashes.
%
This job has been greatly alleviated since the advent of
\LUATEX-Fonts, prior to which the individual dependencies had to be
manually spotted and extracted from the \CONTEXT source code in a
complicated and error-prone fashion.

Below is a commented list of the files distributed with
\identifier{luaotfload} in one way or the other.
%
See figure \ref{file-graph} on page \pageref{file-graph} for a
graphical representation of the dependencies.
%
From \LUATEX-Fonts, only the file \fileent{luatex-fonts-merged.lua}
has been imported as \fileent{luaotfload-fontloader.lua}.
%
It is generated by \fileent{mtx-package}, a \LUA source code merging
too developed by Hans Hagen.\footnote{%
  \fileent{mtx-package} is
  \hyperlink [part of \CONTEXT]{http://repo.or.cz/w/context.git/blob_plain/refs/heads/origin:/scripts/context/lua/mtx-package.lua}
  and requires \fileent{mtxrun}.
  Run
  \inlinecode {mtxrun --script package --help}
  to display further information.
  For the actual merging code see the file
  \fileent{util-mrg.lua} that is part of \CONTEXT.
}
It houses several \LUA files that can be classed in three
categories.

\begindefinitions
    \normalitem \emphasis{\LUA utility libraries}, a subset
                of what is provided by the \identifier{lualibs}
                package.

                \begindoublecolumns
                  \begindefinitions
                      \altitem{l-lua.lua}       \altitem{l-lpeg.lua}
                      \altitem{l-function.lua}  \altitem{l-string.lua}
                      \altitem{l-table.lua}     \altitem{l-io.lua}
                      \altitem{l-file.lua}      \altitem{l-boolean.lua}
                      \altitem{l-math.lua}      \altitem{util-str.lua}
                  \enddefinitions
                \enddoublecolumns

    \normalitem The \emphasis{font loader} itself.
                These files have been written for
                \LUATEX-Fonts and they are distributed along
                with \identifier{luaotfload}.
                \begindoublecolumns
                  \begindefinitions
                    \altitem{luatex-basics-gen.lua}
                    \altitem{luatex-basics-nod.lua}
                    \altitem{luatex-fonts-enc.lua}
                    \altitem{luatex-fonts-syn.lua}
                    \altitem{luatex-fonts-tfm.lua}
                    \altitem{luatex-fonts-chr.lua}
                    \altitem{luatex-fonts-lua.lua}
                    \altitem{luatex-fonts-inj.lua}
                    \altitem{luatex-fonts-otn.lua}
                    \altitem{luatex-fonts-def.lua}
                    \altitem{luatex-fonts-ext.lua}
                    \altitem{luatex-fonts-cbk.lua}
                  \enddefinitions
                \enddoublecolumns

    \normalitem Code related to \emphasis{font handling and
                node processing}, taken directly from
                \CONTEXT.
                \begindoublecolumns
                  \begindefinitions
                    \altitem{data-con.lua} \altitem{font-ini.lua}
                    \altitem{font-con.lua} \altitem{font-cid.lua}
                    \altitem{font-map.lua} \altitem{font-oti.lua}
                    \altitem{font-otf.lua} \altitem{font-otb.lua}
                    \altitem{font-ota.lua} \altitem{font-def.lua}
                    \altitem{font-otp.lua}
                  \enddefinitions
                \enddoublecolumns
\enddefinitions

Note that if \identifier{luaotfload} cannot locate the
merged file, it will load the individual \LUA libraries
instead.
%
Their names remain the same as in \CONTEXT (without the
\inlinecode {otfl}-prefix) since we imported the relevant section of
\fileent{luatex-fonts.lua} unmodified into \fileent{luaotfload-main.lua}.
Thus if you prefer running bleeding edge code from the
\CONTEXT beta, all you have to do is remove
\fileent{luaotfload-merged.lua} from the search path.

Also, the merged file at some point loads the Adobe Glyph List from a
\LUA table that is contained in \fileent{luaotfload-glyphlist.lua},
which is automatically generated by the script
\fileent{mkglyphlist}.\footnote{%
  See \fileent{luaotfload-font-enc.lua}.
  The hard-coded file name is why we have to replace the procedure
  that loads the file in \fileent{luaotfload-override.lua}.
}
There is a make target \identifier{glyphs} that will create a fresh
glyph list so we don’t need to import it from \CONTEXT any longer.

In addition to these, \identifier{luaotfload} requires a number of
files not contained in the merge. Some of these have no equivalent in
\LUATEX-Fonts or \CONTEXT, some were taken unmodified from the latter.


\beginfilelist
    \altitem {luaotfload-features.lua}     font feature handling;
                                           incorporates some of the code from
                                           \fileent{font-otc} from \CONTEXT;
    \altitem {luaotfload-override.lua}     overrides the \CONTEXT logging
                                           functionality.
    \altitem {luaotfload-loaders.lua}      registers the \OpenType
                                           font reader as handler for
                                           Postscript fonts
                                           (\abbrev{pfa}, \abbrev{pfb}).
    \altitem {luaotfload-parsers.lua}      various \abbrev{lpeg}-based parsers.
    \altitem {luaotfload-database.lua}     font names database.
    \altitem {luaotfload-colors.lua}       color handling.
    \altitem {luaotfload-auxiliary.lua}    access to internal functionality
                                           for package authors
                                           (proposals for additions welcome).
    \altitem {luaotfload-letterspace.lua}  font-based letterspacing.
\endfilelist

\beginfigurefloat
  {file-graph}
  {Schematic of the files in \identifier{Luaotfload}}
  \includegraphics[width=\textwidth]{filegraph.pdf}
\endfigurefloat

\endsection

%%%%%%%%%%%%%%%%%%%%%%%%%%%%%%%%%%%%%%%%%%%%%%%%%%%%%%%%%%%%%%%%%%%%%%%%%%%%%%%
\beginsection {Auxiliary Functions}
%%%%%%%%%%%%%%%%%%%%%%%%%%%%%%%%%%%%%%%%%%%%%%%%%%%%%%%%%%%%%%%%%%%%%%%%%%%%%%%

With release version 2.2, \identifier{luaotfload} received
additional functions for package authors to call from outside
(see the file \fileent{luaotfload-auxiliary.lua} for details).
%
The purpose of this addition twofold.
%
Firstly, \identifier{luaotfload} failed to provide a stable interface
to internals in the past which resulted in an unmanageable situation
of different packages abusing the raw access to font objects by means
of the \luafunction{patch_font} callback.
%
When the structure of the font object changed due to an update, all
of these imploded and several packages had to be fixed while
simultaneously providing fallbacks for earlier versions.
%
Now the patching is done on the \identifier{luaotfload} side and can
be adapted with future modifications to font objects without touching
the packages that depend on it.
%
Second, some the capabilities of the font loader and the names
database are not immediately relevant in \identifier{luaotfload}
itself but might nevertheless be of great value to package authors or
end users.

Note that the current interface is not yet set in stone and the
development team is open to suggestions for improvements or
additions.

\beginsubsection {Callback Functions}

The \luafunction{patch_font} callback is inserted in the wrapper
\identifier{luaotfload} provides for the font definition callback
(see below, page \pageref{define-font}).
%
At this place it allows manipulating the font object immediately after
the font loader is done creating it.
%
For a short demonstration of its usefulness, here is a snippet that
writes an entire font object to the file \fileent{fontdump.lua}:

\beginlisting
  \input luaotfload.sty
  \directlua{
    local dumpfile    = "fontdump.lua"
    local dump_font   = function (tfmdata)
      local data = table.serialize(tfmdata)
      io.savedata(dumpfile, data)
    end

    luatexbase.add_to_callback(
      "luaotfload.patch_font",
      dump_font,
      "my_private_callbacks.dump_font"
    )
  }
  \font\dumpme=name:Iwona
  \bye
\endlisting

\emphasis{Beware}: this creates a Lua file of around 150,000 lines of
code, taking up 3~\abbrev{mb} of disk space.
%
By inspecting the output you can get a first impression of how a font
is structured in \LUATEX’s memory, what elements it is composed of,
and in what ways it can be rearranged.

\beginsubsubsection {Compatibility with Earlier Versions}

As has been touched on in the preface to this section, the structure
of the object as returned by the fontloader underwent rather drastic
changes during different stages of its development, and not all
packages that made use of font patching have kept up with every one
of it.
%
To ensure compatibility with these as well as older versions of
some packages, \identifier{luaotfload} sets up copies of or references
to data in the font table where it used to be located.
%
For instance, important parameters like the requested point size, the
units factor, and the font name have again been made accessible from
the toplevel of the table even though they were migrated to different
subtables in the meantime.

\endsubsubsection

\beginsubsubsection{Patches}

These are mostly concerned with establishing compatibility with \XETEX.

\beginfunctionlist

  \altitem  {set_sscale_dimens}
            Calculate \texmacro{fontdimen}s 10 and 11 to emulate \XETEX.

  \altitem  {set_capheight}
            Calculates \texmacro{fontdimen} 8 like \XETEX.

  \altitem  {patch_cambria_domh}
            Correct some values of the font \emphasis{Cambria Math}.

\endfunctionlist

\endsubsection

\beginsubsection {Package Author’s Interface}

As \LUATEX release 1.0 is nearing, the demand for a reliable interface
for package authors increases.

\endsubsubsection

\beginsubsubsection{Font Properties}

Below functions mostly concern querying the different components of a
font like for instance the glyphs it contains, or what font features
are defined for which scripts.

\beginfunctionlist

  \altitem  {aux.font_has_glyph (id : int, index : int)}
            Predicate that returns true if the font \luafunction{id}
            has glyph \luafunction{index}.

  \altitem  {aux.slot_of_name(name : string)}
            Translates an Adobe Glyph name to the corresponding glyph
            slot.

  \altitem  {aux.name_of_slot(slot : int)}
            The inverse of \luafunction{slot_of_name}; note that this
            might be incomplete as multiple glyph names may map to the
            same codepoint, only one of which is returned by
            \luafunction{name_of_slot}.

  \altitem  {aux.provides_script(id : int, script : string)}
            Test if a font supports \luafunction{script}.

  \altitem  {aux.provides_language(id : int, script : string, language : string)}
            Test if a font defines \luafunction{language} for a given
            \luafunction{script}.

  \altitem  {aux.provides_feature(id : int, script : string,
             language : string, feature : string)}
            Test if a font defines \luafunction{feature} for
            \luafunction{language} for a given \luafunction{script}.

  \altitem  {aux.get_math_dimension(id : int, dimension : string)}
            Get the dimension \luafunction{dimension} of font \luafunction{id}.

  \altitem  {aux.sprint_math_dimension(id : int, dimension : string)}
            Same as \luafunction{get_math_dimension()}, but output the value
            in scaled points at the \TEX end.

\endfunctionlist

\endsubsubsection

%% not implemented, may come back later
% \beginsubsubsection{Database}
%
% \beginfunctionlist
%   \altitem  {aux.scan_external_dir(dir : string)}
%             Include fonts in directory \luafunction{dir} in font lookups without
%             adding them to the database.
%
% \endfunctionlist
%
% \endsubsubsection

\endsubsection
\endsection

%%%%%%%%%%%%%%%%%%%%%%%%%%%%%%%%%%%%%%%%%%%%%%%%%%%%%%%%%%%%%%%%%%%%%%%%%%%%%%%
\beginsection {Troubleshooting}
%%%%%%%%%%%%%%%%%%%%%%%%%%%%%%%%%%%%%%%%%%%%%%%%%%%%%%%%%%%%%%%%%%%%%%%%%%%%%%%

\beginsubsection {Database Generation}

If you encounter problems with some fonts, please first update to the
latest version of this package before reporting a bug, as
\identifier{luaotfload} is under active development and still a moving
target.
%
The development takes place on \identifier{github} at
\hyperlink {https://github.com/lualatex/luaotfload} where there is an issue
tracker for submitting bug reports, feature requests and the likes
requests and the likes.

Bug reports are more likely to be addressed if they contain the output
of

\beginlisting
    luaotfload-tool --diagnose=environment,files,permissions
\endlisting

\noindent Consult the man page for a description of these options.

Errors during database generation can be traced by increasing the
verbosity level and redirecting log output to \fileent{stdout}:

\beginlisting
    luaotfload-tool -fuvvv --log=stdout
\endlisting

\noindent or to a file in \fileent{/tmp}:

\beginlisting
    luaotfload-tool -fuvvv --log=file
\endlisting

\noindent In the latter case, invoke the \inlinecode {tail(1)} utility on the
file for live monitoring of the progress.

If database generation fails, the font last printed to the terminal or
log file is likely to be the culprit.
%
Please specify it when reporting a bug, and blacklist it for the time
being (see above, page \pageref{font-blacklist}).

\endsubsection

\beginsubsection {Font Features}

A common problem is the lack of features for some
\OpenType fonts even when specified.
%
This can be related to the fact that some fonts do not provide features
for the \inlinecode {dflt} script (see above on page \pageref{script-tag}),
which is the default one in this package.
%
If this happens, assigning a noth script when the font is defined should
fix it.
%
For example with \inlinecode {latn}:

\beginlisting
    \font\test=file:MyFont.otf:script=latn;+liga;
\endlisting

You can get a list of features that a font defines for scripts and
languages by querying it in \fileent{luaotfload-tool}:

\beginlisting
    luaotfload-tool --find="Iwona" --inspect
\endlisting

\endsubsection

\beginsubsection {\LUATEX Programming}

Another strategy that helps avoiding problems is to not access raw
\LUATEX internals directly.
%
Some of them, even though they are dangerous to access, have not been
overridden or disabled.
%
Thus, whenever possible prefer the functions in the \luafunction{aux}
namespace over direct manipulation of font objects. For example, raw
access to the \luafunction{font.fonts} table like:

\beginlisting
    local somefont = font.fonts[2]
\endlisting

\noindent can render already defined fonts unusable.
%
Instead, the function \luafunction{font.getfont()} should be used
because it has been replaced by a safe variant.

However, \luafunction{font.getfont()} only covers fonts handled by the
font loader, e.~g. \identifier{OpenType} and \identifier{TrueType}
fonts, but not \abbrev{tfm} or \abbrev{ofm}.
%
Should you absolutely require access to all fonts known to \LUATEX,
including the virtual and autogenerated ones, then you need to query
both \luafunction{font.getfont()} and \luafunction{font.fonts}.
%
In this case, best define you own accessor:

\beginlisting
    local unsafe_getfont = function (id)
        local tfmdata = font.getfont (id)
        if not tfmdata then
            tfmdata = font.fonts[id]
        end
        return tfmdata
    end

    --- use like getfont()
    local somefont = unsafe_getfont (2)
\endlisting

\endsubsection
\endsection

\clearpage
%%%%%%%%%%%%%%%%%%%%%%%%%%%%%%%%%%%%%%%%%%%%%%%%%%%%%%%%%%%%%%%%%%%%%%%%%%%%%%%
\beginsection {The GNU GPL License v2}
%%%%%%%%%%%%%%%%%%%%%%%%%%%%%%%%%%%%%%%%%%%%%%%%%%%%%%%%%%%%%%%%%%%%%%%%%%%%%%%

The GPL requires the complete license text to be distributed along
with the code. I recommend the canonical source, instead:
\hyperlink {http://www.gnu.org/licenses/old-licenses/gpl-2.0.html}.
But if you insist on an included copy, here it is.
You might want to zoom in.

\newsavebox{\gpl}
\begin{lrbox}{\gpl}
\begin{minipage}{3\textwidth}
\columnsep=3\columnsep
\begintriplecolumns
\begincentered
  {\Large GNU GENERAL PUBLIC LICENSE\par}
  \bigskip
  {Version 2, June 1991}

  {\parindent 0in

  Copyright \textcopyright\ 1989, 1991 Free Software Foundation, Inc.

  \bigskip

  51 Franklin Street, Fifth Floor, Boston, MA  02110-1301, USA

  \bigskip

  Everyone is permitted to copy and distribute verbatim copies
  of this license document, but changing it is not allowed.
  }

  {\bf\large Preamble}
\endcentered


The licenses for most software are designed to take away your freedom to
share and change it.  By contrast, the GNU General Public License is
intended to guarantee your freedom to share and change free software---to
make sure the software is free for all its users.  This General Public
License applies to most of the Free Software Foundation's software and to
any other program whose authors commit to using it.  (Some other Free
Software Foundation software is covered by the GNU Library General Public
License instead.)  You can apply it to your programs, too.

When we speak of free software, we are referring to freedom, not price.
Our General Public Licenses are designed to make sure that you have the
freedom to distribute copies of free software (and charge for this service
if you wish), that you receive source code or can get it if you want it,
that you can change the software or use pieces of it in new free programs;
and that you know you can do these things.

To protect your rights, we need to make restrictions that forbid anyone to
deny you these rights or to ask you to surrender the rights.  These
restrictions translate to certain responsibilities for you if you
distribute copies of the software, or if you modify it.

For example, if you distribute copies of such a program, whether gratis or
for a fee, you must give the recipients all the rights that you have.  You
must make sure that they, too, receive or can get the source code.  And
you must show them these terms so they know their rights.

We protect your rights with two steps: (1) copyright the software, and (2)
offer you this license which gives you legal permission to copy,
distribute and/or modify the software.

Also, for each author's protection and ours, we want to make certain that
everyone understands that there is no warranty for this free software.  If
the software is modified by someone else and passed on, we want its
recipients to know that what they have is not the original, so that any
problems introduced by others will not reflect on the original authors'
reputations.

Finally, any free program is threatened constantly by software patents.
We wish to avoid the danger that redistributors of a free program will
individually obtain patent licenses, in effect making the program
proprietary.  To prevent this, we have made it clear that any patent must
be licensed for everyone's free use or not licensed at all.

The precise terms and conditions for copying, distribution and
modification follow.

\begincentered
  {\Large \sc Terms and Conditions For Copying, Distribution and
   Modification}
\endcentered

\beginenumeration
\item
This License applies to any program or other work which contains a notice
placed by the copyright holder saying it may be distributed under the
terms of this General Public License.  The ``Program'', below, refers to
any such program or work, and a ``work based on the Program'' means either
the Program or any derivative work under copyright law: that is to say, a
work containing the Program or a portion of it, either verbatim or with
modifications and/or translated into another language.  (Hereinafter,
translation is included without limitation in the term ``modification''.)
Each licensee is addressed as ``you''.

Activities other than copying, distribution and modification are not
covered by this License; they are outside its scope.  The act of
running the Program is not restricted, and the output from the Program
is covered only if its contents constitute a work based on the
Program (independent of having been made by running the Program).
Whether that is true depends on what the Program does.

\item You may copy and distribute verbatim copies of the Program's source
  code as you receive it, in any medium, provided that you conspicuously
  and appropriately publish on each copy an appropriate copyright notice
  and disclaimer of warranty; keep intact all the notices that refer to
  this License and to the absence of any warranty; and give any other
  recipients of the Program a copy of this License along with the Program.

You may charge a fee for the physical act of transferring a copy, and you
may at your option offer warranty protection in exchange for a fee.

\item
You may modify your copy or copies of the Program or any portion
of it, thus forming a work based on the Program, and copy and
distribute such modifications or work under the terms of Section 1
above, provided that you also meet all of these conditions:

\beginenumeration

\item
You must cause the modified files to carry prominent notices stating that
you changed the files and the date of any change.

\item
You must cause any work that you distribute or publish, that in
whole or in part contains or is derived from the Program or any
part thereof, to be licensed as a whole at no charge to all third
parties under the terms of this License.

\item
If the modified program normally reads commands interactively
when run, you must cause it, when started running for such
interactive use in the most ordinary way, to print or display an
announcement including an appropriate copyright notice and a
notice that there is no warranty (or else, saying that you provide
a warranty) and that users may redistribute the program under
these conditions, and telling the user how to view a copy of this
License.  (Exception: if the Program itself is interactive but
does not normally print such an announcement, your work based on
the Program is not required to print an announcement.)

\endenumeration


These requirements apply to the modified work as a whole.  If
identifiable sections of that work are not derived from the Program,
and can be reasonably considered independent and separate works in
themselves, then this License, and its terms, do not apply to those
sections when you distribute them as separate works.  But when you
distribute the same sections as part of a whole which is a work based
on the Program, the distribution of the whole must be on the terms of
this License, whose permissions for other licensees extend to the
entire whole, and thus to each and every part regardless of who wrote it.

Thus, it is not the intent of this section to claim rights or contest
your rights to work written entirely by you; rather, the intent is to
exercise the right to control the distribution of derivative or
collective works based on the Program.

In addition, mere aggregation of another work not based on the Program
with the Program (or with a work based on the Program) on a volume of
a storage or distribution medium does not bring the other work under
the scope of this License.

\item
You may copy and distribute the Program (or a work based on it,
under Section 2) in object code or executable form under the terms of
Sections 1 and 2 above provided that you also do one of the following:

\beginenumeration

\item

Accompany it with the complete corresponding machine-readable
source code, which must be distributed under the terms of Sections
1 and 2 above on a medium customarily used for software interchange; or,

\item

Accompany it with a written offer, valid for at least three
years, to give any third party, for a charge no more than your
cost of physically performing source distribution, a complete
machine-readable copy of the corresponding source code, to be
distributed under the terms of Sections 1 and 2 above on a medium
customarily used for software interchange; or,

\item

Accompany it with the information you received as to the offer
to distribute corresponding source code.  (This alternative is
allowed only for noncommercial distribution and only if you
received the program in object code or executable form with such
an offer, in accord with Subsection b above.)

\endenumeration


The source code for a work means the preferred form of the work for
making modifications to it.  For an executable work, complete source
code means all the source code for all modules it contains, plus any
associated interface definition files, plus the scripts used to
control compilation and installation of the executable.  However, as a
special exception, the source code distributed need not include
anything that is normally distributed (in either source or binary
form) with the major components (compiler, kernel, and so on) of the
operating system on which the executable runs, unless that component
itself accompanies the executable.

If distribution of executable or object code is made by offering
access to copy from a designated place, then offering equivalent
access to copy the source code from the same place counts as
distribution of the source code, even though third parties are not
compelled to copy the source along with the object code.

\item
You may not copy, modify, sublicense, or distribute the Program
except as expressly provided under this License.  Any attempt
otherwise to copy, modify, sublicense or distribute the Program is
void, and will automatically terminate your rights under this License.
However, parties who have received copies, or rights, from you under
this License will not have their licenses terminated so long as such
parties remain in full compliance.

\item
You are not required to accept this License, since you have not
signed it.  However, nothing else grants you permission to modify or
distribute the Program or its derivative works.  These actions are
prohibited by law if you do not accept this License.  Therefore, by
modifying or distributing the Program (or any work based on the
Program), you indicate your acceptance of this License to do so, and
all its terms and conditions for copying, distributing or modifying
the Program or works based on it.

\item
Each time you redistribute the Program (or any work based on the
Program), the recipient automatically receives a license from the
original licensor to copy, distribute or modify the Program subject to
these terms and conditions.  You may not impose any further
restrictions on the recipients' exercise of the rights granted herein.
You are not responsible for enforcing compliance by third parties to
this License.

\item
If, as a consequence of a court judgment or allegation of patent
infringement or for any other reason (not limited to patent issues),
conditions are imposed on you (whether by court order, agreement or
otherwise) that contradict the conditions of this License, they do not
excuse you from the conditions of this License.  If you cannot
distribute so as to satisfy simultaneously your obligations under this
License and any other pertinent obligations, then as a consequence you
may not distribute the Program at all.  For example, if a patent
license would not permit royalty-free redistribution of the Program by
all those who receive copies directly or indirectly through you, then
the only way you could satisfy both it and this License would be to
refrain entirely from distribution of the Program.

If any portion of this section is held invalid or unenforceable under
any particular circumstance, the balance of the section is intended to
apply and the section as a whole is intended to apply in other
circumstances.

It is not the purpose of this section to induce you to infringe any
patents or other property right claims or to contest validity of any
such claims; this section has the sole purpose of protecting the
integrity of the free software distribution system, which is
implemented by public license practices.  Many people have made
generous contributions to the wide range of software distributed
through that system in reliance on consistent application of that
system; it is up to the author/donor to decide if he or she is willing
to distribute software through any other system and a licensee cannot
impose that choice.

This section is intended to make thoroughly clear what is believed to
be a consequence of the rest of this License.

\item
If the distribution and/or use of the Program is restricted in
certain countries either by patents or by copyrighted interfaces, the
original copyright holder who places the Program under this License
may add an explicit geographical distribution limitation excluding
those countries, so that distribution is permitted only in or among
countries not thus excluded.  In such case, this License incorporates
the limitation as if written in the body of this License.

\item
The Free Software Foundation may publish revised and/or new versions
of the General Public License from time to time.  Such new versions will
be similar in spirit to the present version, but may differ in detail to
address new problems or concerns.

Each version is given a distinguishing version number.  If the Program
specifies a version number of this License which applies to it and ``any
later version'', you have the option of following the terms and conditions
either of that version or of any later version published by the Free
Software Foundation.  If the Program does not specify a version number of
this License, you may choose any version ever published by the Free Software
Foundation.

\item
If you wish to incorporate parts of the Program into other free
programs whose distribution conditions are different, write to the author
to ask for permission.  For software which is copyrighted by the Free
Software Foundation, write to the Free Software Foundation; we sometimes
make exceptions for this.  Our decision will be guided by the two goals
of preserving the free status of all derivatives of our free software and
of promoting the sharing and reuse of software generally.

\begincentered
  {\Large\sc No Warranty}
\endcentered

\item
{\sc Because the program is licensed free of charge, there is no warranty
for the program, to the extent permitted by applicable law.  Except when
otherwise stated in writing the copyright holders and/or other parties
provide the program ``as is'' without warranty of any kind, either expressed
or implied, including, but not limited to, the implied warranties of
merchantability and fitness for a particular purpose.  The entire risk as
to the quality and performance of the program is with you.  Should the
program prove defective, you assume the cost of all necessary servicing,
repair or correction.}

\item
{\sc In no event unless required by applicable law or agreed to in writing
will any copyright holder, or any other party who may modify and/or
redistribute the program as permitted above, be liable to you for damages,
including any general, special, incidental or consequential damages arising
out of the use or inability to use the program (including but not limited
to loss of data or data being rendered inaccurate or losses sustained by
you or third parties or a failure of the program to operate with any other
programs), even if such holder or other party has been advised of the
possibility of such damages.}

\endenumeration


\begincentered
  {\Large\sc End of Terms and Conditions}
\endcentered


\pagebreak[2]

%%%%%%%%%%%%%%%%%%%%%%%%%%%%%%%%%%%%%%%%%%%%%%%%%%%%%%%%%%%%%%%%%%%%%%%%%%%%%%%
\fakesection {Appendix: How to Apply These Terms to Your New Programs}
%%%%%%%%%%%%%%%%%%%%%%%%%%%%%%%%%%%%%%%%%%%%%%%%%%%%%%%%%%%%%%%%%%%%%%%%%%%%%%%

If you develop a new program, and you want it to be of the greatest
possible use to the public, the best way to achieve this is to make it
free software which everyone can redistribute and change under these
terms.

  To do so, attach the following notices to the program.  It is safest to
  attach them to the start of each source file to most effectively convey
  the exclusion of warranty; and each file should have at least the
  ``copyright'' line and a pointer to where the full notice is found.

\beginnarrower
  one line to give the program's name and a brief idea of what it does. \\
  Copyright (C) yyyy  name of author \\

  This program is free software; you can redistribute it and/or modify
  it under the terms of the GNU General Public License as published by
  the Free Software Foundation; either version 2 of the License, or
  (at your option) any later version.

  This program is distributed in the hope that it will be useful,
  but WITHOUT ANY WARRANTY; without even the implied warranty of
  MERCHANTABILITY or FITNESS FOR A PARTICULAR PURPOSE.  See the
  GNU General Public License for more details.

  You should have received a copy of the GNU General Public License
  along with this program; if not, write to the Free Software
  Foundation, Inc., 51 Franklin Street, Fifth Floor, Boston, MA  02110-1301, USA.
\endnarrower

Also add information on how to contact you by electronic and paper mail.

If the program is interactive, make it output a short notice like this
when it starts in an interactive mode:

\beginnarrower
  Gnomovision version 69, Copyright (C) yyyy  name of author \\
  Gnomovision comes with ABSOLUTELY NO WARRANTY; for details type `show w'. \\
  This is free software, and you are welcome to redistribute it
  under certain conditions; type `show c' for details.
\endnarrower


The hypothetical commands {\tt show w} and {\tt show c} should show the
appropriate parts of the General Public License.  Of course, the commands
you use may be called something other than {\tt show w} and {\tt show c};
they could even be mouse-clicks or menu items---whatever suits your
program.

You should also get your employer (if you work as a programmer) or your
school, if any, to sign a ``copyright disclaimer'' for the program, if
necessary.  Here is a sample; alter the names:

\beginnarrower
Yoyodyne, Inc., hereby disclaims all copyright interest in the program \\
`Gnomovision' (which makes passes at compilers) written by James Hacker. \\

signature of Ty Coon, 1 April 1989 \\
Ty Coon, President of Vice
\endnarrower


This General Public License does not permit incorporating your program
into proprietary programs.  If your program is a subroutine library, you
may consider it more useful to permit linking proprietary applications
with the library.  If this is what you want to do, use the GNU Library
General Public License instead of this License.

\endtriplecolumns
\end{minipage}
\end{lrbox}

\begincentered
  \scalebox{0.33}{\usebox{\gpl}}
\endcentered

\endsection

\endinput


\end {document}


